% !TEX root = categorical.tex
% !LPiL preamble = ../dPreamble.tex
% !LPiL postamble = ../dPostamble.tex

\lpilSection{fp-diag-cat}{The basics of drawing Categorical diagrams}

We will need to draw lots of Categorical diagrams.

Here is a first simple square (using MetaFun/Nodes).

\begin{lpil:metaFun}{simpleSquare_mp.tex}
% !TEX root    = simpleSquare_mp.tex
% !TEX program = metafun
\defineframed
  [nodeSmall]
  [node]
  [foregroundstyle=small]
\startMPpage
clearnodepath ;
path p ; p = fullsquare scaled 3cm ;
nodepath = p ;
draw node(0,"\node{$G(X)$}") ;
draw node(1,"\node{$G(Y)$}") ;
draw node(2,"\node{$F(Y)$}") ;
draw node(3,"\node{$F(X)$}") ;
drawarrow fromto.bot(0, 0, 1, "\nodeSmall{$G(f)$}") ;
drawarrow fromto.top(0, 3, 2, "\nodeSmall{$F(f)$}") ;
drawarrow fromto.rt (0, 2, 1, "\nodeSmall{$\eta_Y$}") ;
drawarrow fromto.lft(0, 3, 0, "\nodeSmall{$\eta_X$}") ;
\stopMPpage
\end{lpil:metaFun}

\includeLpilDiagram{simpleSquare_mp}
